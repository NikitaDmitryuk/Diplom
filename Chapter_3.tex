\newpage
\begin{center}
\textbf{\large ЗАКЛЮЧЕНИЕ}
\end{center}
\refstepcounter{chapter}


% \section*{}
\addcontentsline{toc}{chapter}{ЗАКЛЮЧЕНИЕ}
Основные результаты бакалаврской квалификационной работы:
\begin{enumerate}
    \item В работе продемонстрированная актуальность метода изучения молекулярных систем с использованием построения диаграммы Вороного, который, используя только координаты частиц в разные моменты времени, позволяет узнать некоторые термодинамические параметры и параметры переноса в веществе.
    \item Проведена модернизация алгоритма классификации частиц на фазы, которая позволяет улучшить качество классификации и убрать артефакты из статистики распределения плотности частиц.
    \item Проведены серии моделирований обобщенного потенциала взаимодействия Леннарда--Джонса с различным дальнодействием притяжения для определения его роли в расположении критических, тройных точек и линий Видома на фазовых диаграммах вещества. 
    \item Реализован программный пакет для изучения диффузии с разрешением отдельных частиц, и с его помощью установлена роль дальнодействия притяжения на подвижность в веществе. 
\end{enumerate}

% В данной работе рассмотрены методы изучения молекулярных систем, которые, используя только координаты частиц в разные моменты времени, позволяют узнать некоторые термодинамические параметры системы.

% Было показано, каким образом можно эффективно классифицировать частицы на газ, конденсат и поверхность, с помощью разбиение системы на ячейки вороного. Проведена модернизация алгоритма классификации, которая позволяет улучшить качество распознавания фаз, и убрать артефакты из статистики распределения плотности частиц.

% Разработан и автоматизирован метод, позволяющий определять критические точки в веществе с удовлетворительной точностью.

% Установлена роль притяжения в расположении тройных и критических точек, а также точек с наибольшими флуктуациями плотности в системе.

% Предложен способ определения сжимаемости и скорости звука в веществе, используя только распределение плотностей ячеек вороного, а так же способ определения линии Видома для плотности, с помощью моментов величины плотности вещества.

% Установлено влияние дальнодействия притяжения на параметры переноса частиц в веществе, и их температурные зависимости. 
% Предложен способ классификации системы на кристалл, жидкость и сверхкритическую жидкость с помощью нейронных сетей, по параметрам скорости звука и мобильности в веществе.

\newpage
\begin{center}
\textbf{ВЫВОДЫ РАБОТЫ}
\end{center}
\refstepcounter{chapter}


% \section*{}
\addcontentsline{toc}{chapter}{ВЫВОДЫ РАБОТЫ}

В \textbf{главе 2} были рассмотрены методы изучения молекулярных систем, которые, используя только координаты частиц в разные моменты времени, позволяют узнать некоторые термодинамические параметры системы, а так же довольно точно определять критические точки в веществах. 

Было показано, как можно эффективно классифицировать частицы на газ, конденсат и поверхность, с помощью разбиение системы на ячейки вороного. Была проведена модернизация алгоритма классификации, которая рассмотрена в разделе \ref{C2_2}.

Так же было показано, каким образом наиболее точно можно определить критические точки в системах, и каким образом это можно автоматизировать. 

Установлена роль притяжения на фазовые диаграммы веществ для потенциалов, исследуемых в данной работе.

Представлен способ определения сжимаемости и скорости звука в веществе, используя только распределение плотностей ячеек вороного, а так же способ определения линии Уидома для плотности, с помощью моментов величины плотности вещества.

В \textbf{главе 3} был рассмотрен способ изучения диффузии методом молекулярной динамики.
Показано, что температурная зависимость мобильности частиц зависит линейно от температуры в промежутке между тройной и критической точкой. Так же установлено, что данный метод можно использовать для достаточно точного определения тройной точки в веществе. Кроме того, была рассмотрена связь термодинамических свойств системы с параметрами переноса в веществе. 

\newpage
\begin{center}
\textbf{ВЫВОДЫ РАБОТЫ}
\end{center}
\refstepcounter{chapter}


% \section*{}
\addcontentsline{toc}{chapter}{ВЫВОДЫ РАБОТЫ}

В данной работе в \textbf{главе 2} были рассмотрены методы изучения молекулярных систем, которые, используя только координаты частиц в разные моменты времени, позволяют узнать некоторые термодинамические параметры системы, а так же довольно точно определять критические точки в веществах. 

Было показано, каким образом можно эффективно классифицировать частицы на газ, конденсат и поверхность, с помощью разбиение системы на ячейки вороного. Была проведена модернизация алгоритма классификации, которая позволяет улучшить качество распознавания фаз, и убрать артефакты из статистики распределений по ячейкам различных величин.

Разработан и автоматизирован метод, позволяющий определять критические точки в веществе с удовлетворительной точностью.

Установлена роль притяжения в расположении тройных и критических точек, а также точек с наибольшими флуктуациями плотности в системе.

Предложен способ определения сжимаемости и скорости звука в веществе, используя только распределение плотностей ячеек вороного, а так же способ определения линии Видома для плотности, с помощью моментов величины плотности вещества.

Кроме того, в данной работе было выяснено влияние дальнодействия притяжения на подвижность частиц в веществе, и их температурные зависимости. Предложен способ классификации системы на кристалл, жидкость и сверхкритическую жидкость с помощью нейронных сетей, по параметрам скорости звука и мобильности в веществе.

%\chapter*{Введение}
%\addcontentsline{toc}{chapter}{Введение}

\newpage
\begin{center}
\textbf{АННОТАЦИЯ}
\end{center}
%\refstepcounter{chapter}
\addcontentsline{toc}{chapter}{АННОТАЦИЯ}

В данной работе объектом исследования были фазовые диаграммы для систем с обобщенным потенциалом взаимодействия Леннарда-Джонса. Для решения данной задачи были использованы методы молекулярной динамики и пост-обработка с использованием MATLAB и математических и графических библиотек для языка Python.

 В ходе работы были проведены множественные моделирования систем с различными потенциалами взаимодействия, модернизированы методы пост-обработки систем с разрешением отдельных частиц с помощью разбиения на ячейки Вороного. Реализован программный пакет для определения тройных и критических точек в веществе по фазовой диаграмме, а также нахождения температуры наибольших флуктуаций плотности системы.
 
 На основании полученных зависимостей выявлено влияние дальнодействия притяжения в молекулярных системах на термодинамические параметры, такие как скорость звука и сжимаемость, а также положение тройной и критической точки и параметры переноса вещества. 


\onehalfspacing
\setcounter{page}{2}
\renewcommand{\contentsname}{\centerline{\Large{Cодержание}}}
\tableofcontents
\addtocontents{toc}{\protect\thispagestyle{fancy}}
\renewcommand{\contentsname}{\centerline{\Large{Cодержание}}}

\newpage
\begin{center}
\textbf{ВВЕДЕНИЕ}
\end{center}
%\refstepcounter{chapter}
\addcontentsline{toc}{chapter}{ВВЕДЕНИЕ}



\textbf{Актуальность}

Для физики конденсированного состояния большой интерес представляют такие явления, как кристаллизация, плавление, критические явление, а так же их зависимость от микроскопических параметров системы. Понимание влияния этих параметров на термодинамику системы, играет важную роль не только с точки зрения фундаментальных знаний, но и прикладных, например, в материаловедении, промышленности и медицине.

Знание зависимости внешних параметров системы от потенциала взаимодействия, является открытым вопросом в физике мягкой материи. Точное прогнозирование, или хотя бы качественная их оценка, термодинамических параметров вещества, для которого известен состав и внешние условия (например, внешние электрические или магнитные поля), позволят избежать дорогостоящих исследований поведения каждого отдельного вещества. 

Также это открывает возможности для создания новых веществ, удовлетворяющих потребности в определенном фазовом поведении, с нужными температурами плавления или скорости звука в веществе, а также сжимаемости.

\newpage

\textbf{Цель работы} --
установить связь между дальнодействием притяжения в двумерной системе частиц, взаимодействующих посредством обобщенного потенциала Леннарда-Джонса, c фазовой диаграммой, и параметрами переноса.

\textbf{Задачами работы являются:}
\begin{enumerate}
\item Разработка программного комплекса для расчета явлений переноса в $2D$ системах.
\item Разработка методов определения термодинамических свойств системы по распределениям плотностей. 
\item Усовершенствование метода распознавание фаз и построения фазовых диаграмм.
\item Применение разработанных методов на различных потенциалах взаимодействия.
\item Применение наработок для изучения влияния потенциала взаимодействия на различные термодинамические параметры.
\end{enumerate}

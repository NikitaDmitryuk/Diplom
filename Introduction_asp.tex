%\chapter*{Введение}
%\addcontentsline{toc}{chapter}{Введение}

\newpage
\begin{center}
\textbf{ВВЕДЕНИЕ}
\end{center}
%\refstepcounter{chapter}
\addcontentsline{toc}{chapter}{ВВЕДЕНИЕ}



\textbf{Актуальность.}

Для физики конденсированного состояния большой интерес представляют такие явления, как кристаллизация, плавление, критические явление, а так же их зависимость от свойств системы. Понимание влияния этих свойств на систему играет важную роль в материаловедении.

На данный момент эти проблемы решаются с использованием модельных систем, которые позволяют наблюдать в отдельности за каждой частицей в смоделированной системе. Примерами модельных систем служат коллоидные системы и пылевая плазма.



\textbf{Цель работы} --
установить связь между дальнодействием притяжения в двумерной системе частиц, взаимодействующих посредством обобщенного потенциала Леннарда-Джонса, c фазовой диаграммой, и параметрами переноса.

\textbf{Задачи работы:}
\begin{enumerate}
\item Разработка программного комплекса для расчета явлений переноса в $2D$ системах.
\item Разработка методов определения термодинамических свойств системы по распределениям плотностей. 
\item Усовершенствование метода распознавание фаз и построения фазовых диаграмм.
\item Применение разработанных методов на различных потенциалах взаимодействия.
\item Применение наработок для изучения влияния потенциала взаимодействия на различные термодинамические параметры.
\end{enumerate}

\textbf{Научная новизна работы:}
\begin{enumerate}
\item Впервые показано, что термодинамические свойства системы могут быть рассчитаны по распределению статических параметров.

\end{enumerate}

\textbf{Положения, выносимые на защиту:}
\begin{enumerate}
\item Показано, что ...

\item Показано, что ...

\item Показано, что ...

\end{enumerate}

\textbf{Методология и методы исследования.} 
Сформулированные задачи были решены с помощью моделирования систем методами молекулярной динамики, с использованием свободного программного пакета LAMMPS. Пост-обработка результатов выполнена с помощью разработанного программного комплекса на языке MATLAB и Python.


\textbf{Достоверность.} 

\textbf{Личный вклад автора.}

\textbf{Теоретической значимостью.} 

\textbf{Практическая значимость.} 

\textbf{Результат работы.} 

\textbf{Апробация работы.} 

\textbf{Публикации.} 
Основные результаты работы находятся на рецензировании.

\textbf{Структура и объем работы.} 
Научно квалификационная работа состоит из введения, 3 глав и заключения, содержит N страниц, N рисунков.
Список литературы включает N источников.

Во \textbf{введении} кратко обосновывается актуальность работы, формулируется цель, перечисляются положения, выносимых на защиту, указывается научная новизна, достоверность, фундаментальная и практическая значимость результатов работы, личный вклад автора, апробация работы и содержание по главам.


\textbf{Глава 1} является обзорной.
В разделе \ref{C1_1} кратко рассматриваются межмолекулярные взаимодействия.
В разделе \ref{C1_2} рассматриваются взаимодействия коллойдов в присутствии внешних полей.

В разделе \ref{C1_3} рассматриваются параметры переноса в веществе. 


\textbf{Глава 2} посвящена изучению роли дальнодействия притяжения в фазовых диаграммах.


В разделе \ref{C2_1} излагается метод разбиения системы на ячейки вороного.

В разделе \ref{C2_2} демонстрируется применение предложенного подхода на различных потенциалах взаимодействия.

В разделе \ref{C2_3} демонстрируется методы анализа гистограмм распределения плотностей различных потенциалах взаимодействия.

В разделе \ref{C2_4} \textbf{обобщаются основные результаты главы}.

\textbf{Глава 3} посвящена рассмотрению явлений переноса при разных потенциалах взаимодействия.

В разделе \ref{C3_1} рассматриваются методы измерения параметров переноса вещества на различных потенциалах взаимодействия.

В разделе \ref{C3_2} рассматриваются связь термодинамических параметров, и параметров переноса вещества.

В разделе \ref{C3_3} обобщаются основные результаты главы.

В \textbf{общих выводах и заключении} обобщаются основные результаты работы.
%\newpage
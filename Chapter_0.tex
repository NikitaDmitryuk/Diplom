%\chapter*{Введение}
%\addcontentsline{toc}{chapter}{Введение}

\newpage
\begin{center}
\textbf{ГЛАВА 1}\\
\textbf{НАЗВАНИЕ ПЕРВОЙ ГЛАВЫ}
\end{center}
\refstepcounter{chapter}


% \section*{}
\addcontentsline{toc}{chapter}{ГЛАВА 1. Название первой главы}


\section{Межмолекулярные взаимодействия}\label{C1_1}

К межмолекулярным взаимодействиям относятся взаимодействия между молекулами и/или атомами, не приводящие к образованию ковалентных химических связей.

Межмолекулярные взаимодействия имеют электростатическую природу. На больших расстояниях преобладают силы притяжения, которые могут иметь ориентационную, поляризационную и дисперсионную природу.

В случае коллоидных частиц, как правило, из-за разного материала частиц и сольвента возникает притяжение Ван-дер-Ваальса \cite{Yur31, Yur53}

\begin{equation}
\varphi_{\mathrm{vdW}}(r)=-\frac{A_{\mathrm{H}}}{12}\left(\frac{\sigma^{2}}{r^{2}-\sigma^{2}}+\frac{\sigma^{2}}{r^{2}}+2 \ln \frac{r^{2}-\sigma^{2}}{r^{2}}\right)
\end{equation}
где постоянная Хамакера $A_{\mathrm{H}} \propto\left(\frac{\varepsilon_{\mathrm{r}}-1}{\varepsilon_{\mathrm{r}}+1}\right)^{2}$ зависит от относительной диэлектрической проницаемости $\varepsilon_{\mathrm{r}}=\varepsilon_{\mathrm{P}} / \varepsilon_{\mathrm{S}}$. 

Для предотвращения коагуляции в коллоидной системе присутствуют силы отталкивания, которые обусловлены зарядовой или стерической стабилизацией.

Зарядовая стабилизация возникает благодаря взаимному отталкиванию заряженных частиц, в результате накопленного на их поверхности отрицательного заряда, который возникает при диссоциации поверхности и адсорбции ионов. 
В рамках линеаризованной теории Пуассона-Больцмана, взаимодействие Дерягина-Ландау-Фервея-Овербека имеет вид \cite{Yur54}
\begin{equation}
\varphi_{Y}(r)=\left\{\begin{array}{ll}
\infty & r<\sigma, \\
\epsilon_{\mathrm{Y}} \frac{e^{-\kappa(r-\sigma)}}{r / \sigma} & r \geq \sigma,
\end{array}\right.
\end{equation}

где $\kappa=\sqrt{4 \pi \lambda_{\mathrm{B}} n_{\mathrm{ion}}} \equiv \lambda_{\mathrm{D}}^{-1}$ - обратная дебаевская длина экранирования, выраженная через плотность малых ионов $n_{ion}$ и длину Бьеррума $\lambda_{\mathrm{B}}=e^{2} / \varepsilon_{\mathrm{w}} k_{\mathrm{B}} T$. Контактный потенциал записывается как 

\begin{equation}
\epsilon_{\mathrm{Y}}=\frac{Z^{2}}{(1+\kappa \sigma / 2)^{2}} \frac{\lambda_{\mathrm{B}}}{\sigma} k_{\mathrm{B}} T,
\end{equation}
где $Z \equiv Q / e$ зарядовое число коллоида.

Результирующее взаимодействие представляет собой сумму притягивающих и отталкивающих сил:
\begin{equation}
\varphi(r)=\varphi_{Y}(r)+\varphi_{\mathrm{vdW}}(r).
\end{equation}
Вклады данных слагаемых соизмеримы на малом расстоянии между не сильно заряженными частицами.

С ростом заряда частиц, теория Пуассона-Больцмана становится неприменимой вблизи поверхности частицы, однако на дальних расстояния по прежнему имеет форму Юкавы, и при ренормированном зарядом может хорошо описывать потенциал вдали от поверхности \cite{Yur55}. Эффективный насыщенный заряд выражается следующим уравнением \cite{Yur56}
\begin{equation}
Z_{\mathrm{eff}}^{\mathrm{sat}}=(2+\kappa \sigma) \sigma / \lambda_{\mathrm{B}}
\end{equation}
Используя линеаризованную теорию Пуассона - Больцмана с установленным эффективным зарядом, можно объяснить большинство экспериментальных наблюдений \cite{Yur57, Yur58, Yur59}. Это делает теорию Дерягина-Ландау-Фервея-Овербека (ДЛФО) одним из наиболее успешных подходов при описании межчастичных взаимодействий \cite{Yur49, Yur60, Yur61}. 

Помимо зарядовой стабилизации существует так называемая стерическая стабилизация. Она заключается в добавлении в сольвент полимерных молекул, которые оседают на частицах. При сближении частиц, эти полимерные цепи взаимодействуют друг с другом, не давая частицам сблизиться сильнее. 

Часть потенциала взаимодействия, которая соответствует стерической стабилизации, выглядит следующим образом \cite{Yur31}
\begin{equation}
\begin{array}{l}
\varphi_{\text {steric }}(r)=\frac{\pi \sigma}{2} \int_{r-\sigma}^{\infty} d h F(h) \\
\\
F(r)=\frac{\alpha k_{\mathrm{B}} T}{s^{3}}\left[\left(\frac{2 L}{r}\right)^{9 / 4}-\left(\frac{r}{2 L}\right)^{3 / 4}\right], \quad r<2 L
\end{array}
\label{eqStericStabl}
\end{equation}
где $L$ - толщина полимерного слоя, $\alpha$ - численный множитель, определяемый для конкретных полимеров
особенностями взаимодействия между молекулярными цепочками, $s$ - среднее расстояние между привитыми полимерами на поверхности.

В случае наличия этих взаимодействий, суммарный потенциал частиц выражается следующей формулой

\begin{equation}
\varphi(r)=\varphi_{Y}(r)+\varphi_{\mathrm{vdW}}(r)+\varphi_{\mathrm{steric}}(r).
\end{equation}

Для изучения влияния микроскопических свойств на макроскопические свойства, используется так называемый метод молекулярной динамики, который заключается в численном моделировании системы, состоящей из достаточно большого количества частиц, по статистике которых можно судить о макроскопических свойствах. Одним из наиболее популярных модельных потенциалов взаимодействия частиц в таких системах, является потенциал Леннарда - Джонса 
\begin{equation}
U\left(R_{i j}\right)=\varepsilon\left[\left(\frac{R_{0}}{R_{i j}}\right)^{12}-2\left(\frac{R_{0}}{R_{i j}}\right)^{6}\right]=4 \varepsilon\left[\left(\frac{\sigma}{R_{i j}}\right)^{12}-\left(\frac{\sigma}{R_{i j}}\right)^{6}\right], 
\label{eqFullLJ}
\end{equation}
где $\varepsilon$ и $R_0$ - глубина потенциальной ямы и равновесное расстояние между частицами; $R_0 = 2^{1/6}\sigma$.

В то время как зависимость $R^{-6}$ получена теоретически и обусловлена силами Ван-дер-Ваальса, зависимость $R^{-12}$ выбрана из соображений удобства.

Различные физические величины в данных моделированниях удобно выражать через константы моделирования $\sigma, \varepsilon, m$.

Метод молекулярной динамики (МД) позволяет в данной работе выяснить влияние дальнодействия притяжения на термодинамические свойства системы и параметры переноса.

\section{Регулируемые межчастичные взаимодействия}\label{C1_2}



\section{Цели и задачи работы}

\textbf{Цель бакалаврской работы}:

установить связь между дальнодействием притяжения в двумерной системе частиц, взаимодействующих посредством обобщенного потенциала Леннарда-Джонса, и фазовой диаграммой, а также параметров переноса.

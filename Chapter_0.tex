%\chapter*{Введение}
%\addcontentsline{toc}{chapter}{Введение}

\newpage
\begin{center}
\textbf{ГЛАВА 1}\\
\textbf{НАЗВАНИЕ ПЕРВОЙ ГЛАВЫ}
\end{center}
\refstepcounter{chapter}


% \section*{}
\addcontentsline{toc}{chapter}{ГЛАВА 1. Название первой главы}


\section{Межмолекулярные взаимодействия}\label{C1_1}

К межмолекулярным взаимодействиям относятся взаимодействия между молекулами и/или атомами, не приводящие к образованию ковалентных химических связей.

Межмолекулярные взаимодействия имеют электростатическую природу. На больших расстояниях преобладают силы притяжения, которые могут иметь ориентационную, поляризационную и дисперсионную природу.

В случае коллоидных частиц, как правило, из-за разного материала частиц и сольвента возникает притяжение Ван-дер-Ваальса [31, 53!!]

\begin{equation}
\varphi_{\mathrm{vdW}}(r)=-\frac{A_{\mathrm{H}}}{12}\left(\frac{\sigma^{2}}{r^{2}-\sigma^{2}}+\frac{\sigma^{2}}{r^{2}}+2 \ln \frac{r^{2}-\sigma^{2}}{r^{2}}\right)
\end{equation}
где постоянная Хамакера $A_{\mathrm{H}} \propto\left(\frac{\varepsilon_{\mathrm{r}}-1}{\varepsilon_{\mathrm{r}}+1}\right)^{2}$ зависит от относительной диэлектрической проницаемости $\varepsilon_{\mathrm{r}}=\varepsilon_{\mathrm{P}} / \varepsilon_{\mathrm{S}}$. 

Для предотвращения коагуляции в коллоидной системе присутствуют силы отталкивания, которые обусловлены зарядовой или стерической стабилизацией.

Зарядовая стабилизация возникает благодаря взаимному отталкиванию частиц в результате накопленного на их поверхности отрицательного заряда, который возникает в результате диссоциации поверхности и адсорбции ионов. 
Выражение для силы отталкивания заряженных частиц выражается так называемым взаимодействием Дерягина-Ландау-Фервея-Овербека, которое имеет вид [54!!]

\begin{equation}
\varphi_{Y}(r)=\left\{\begin{array}{ll}
\infty & r<\sigma, \\
\epsilon_{\mathrm{Y}} \frac{e^{-\kappa(r-\sigma)}}{r / \sigma} & r \geq \sigma,
\end{array}\right.
\end{equation}

где $\kappa=\sqrt{4 \pi \lambda_{\mathrm{B}} n_{\mathrm{ion}}} \equiv \lambda_{\mathrm{D}}^{-1}$ - обратная дебаевская длина экранирования, выраженная через плотность малых ионов $n_{ion}$ и длину Бьеррума $\lambda_{\mathrm{B}}=e^{2} / \varepsilon_{\mathrm{w}} k_{\mathrm{B}} T$. Контактный потенциал записывается как 

\begin{equation}
\epsilon_{\mathrm{Y}}=\frac{Z^{2}}{(1+\kappa \sigma / 2)^{2}} \frac{\lambda_{\mathrm{B}}}{\sigma} k_{\mathrm{B}} T,
\end{equation}
где $Z \equiv Q / e$ зарядовое число коллоида.

Результирующее взаимодействие представляет собой сумму притягивающих и отталкивающих сил:
\begin{equation}
\varphi(r)=\varphi_{Y}(r)+\varphi_{\mathrm{vdW}}(r).
\end{equation}
Данный потенциал хорошо описывает поведение коллоидных систем.

Для изучения влияния микроскопических свойств на макроскопические свойства, используется так называемый метод молекулярной динамики, который заключается в численном моделировании системы, состоящей из достаточно большого количества частиц, по статистике которых можно судить о макроскопических свойствах. Одним из наиболее популярных модельных потенциалов взаимодействия частиц в таких системах, является потенциал Леннарда - Джонса 
\begin{equation}
U\left(R_{i j}\right)=\varepsilon\left[\left(\frac{R_{0}}{R_{i j}}\right)^{12}-2\left(\frac{R_{0}}{R_{i j}}\right)^{6}\right]=4 \varepsilon\left[\left(\frac{\sigma}{R_{i j}}\right)^{12}-\left(\frac{\sigma}{R_{i j}}\right)^{6}\right], 
\label{eqFullLJ}
\end{equation}
где $\varepsilon$ и $R_0$ - глубина потенциальной ямы и равновесное расстояние между частицами; $R_0 = 2^{1/6}\sigma$.

В то время как зависимость $R^{-6}$ получена теоретически и обусловлена силами Ван-дер-Ваальса, зависимость $R^{-12}$ выбрана из соображений удобства.

Различные физические величины в данных моделированниях удобно выражать через константы моделирования $\sigma, \varepsilon, m$.

Метод молекулярной динамики (МД) позволяет в данной работе выяснить влияние дальнодействия притяжения на термодинамические свойства системы и параметры переноса.

\section{Регулируемые межчастичные взаимодействия}\label{C1_2}

\section{Цели и задачи работы}

\textbf{Цель бакалаврской работы}:

установить связь между дальнодействием притяжения в двумерной системе частиц, взаимодействующих посредством обобщенного потенциала Леннарда-Джонса, и фазовой диаграммой, а также параметров переноса.



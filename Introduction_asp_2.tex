%\chapter*{Введение}
%\addcontentsline{toc}{chapter}{Введение}

\newpage
\begin{center}
\textbf{Введение}
\end{center}
%\refstepcounter{chapter}
\addcontentsline{toc}{chapter}{Введение}



\textbf{Актуальность.}

Открытые и неравновесные системы различного рода встречаются в физике, химии, материаловедении и даже мультиагентных системах или сетях. Понимание механизмов, ответственных за коллективную динамику в таких системах (самоорганизация, диссипативные фазовые переходы и критические явления), имеет важное значение для различных фундаментальных и прикладных задач. 

Невзаимность взаимодействия, т. е. нарушение симметрии действие– реакция для эффективных межчастичных взаимодействий, является характерной чертой для открытых систем. Эффективные взаимодействия в этих системах опосредуются неравновесной средой, примерами которой могут быть: коллоидные суспензии и комплексная плазма, где взаимодействия опосредуются потоками, неравновесные флуктуации, оптические пучки, диффузиофорез и плазменные вейки. 

Последние получаются, если заряженные микрочастицы левитируют в слабоионизированном газе. Благодаря взаимодействию гравитационных и электрических сил частицы могут образовывать однослойные или многослойные структуры, в зависимости от условий эксперимента. На взаимодействие между микрочастицами обычно влияет вертикальный поток плазмы, который генерирует так называемые "плазменные вейки" под каждой частицей. Наличие вейков приводит к невзаимности эффективного взаимодействия, которое зависит от параметров плазменного разряда и может быть настроено в экспериментах. В результате микрочастицы могут  получать энергию из потока плазмы.



\textbf{Цель работы} --
разработка различных подходов к построению фазовых диаграмм в комплексной (пылевой) плазме при различных значениях силы конфайнмента.

\textbf{Задачи работы:}
\begin{enumerate}
\item Систематическое моделирование пылевой (комплексной) плазмы методами МД при различных значениях силы вертикального конфайнмента.
\item
\item
\end{enumerate}

\textbf{Научная новизна работы:}
\begin{enumerate}
\item Впервые показано, что .

\end{enumerate}

\textbf{Положения, выносимые на защиту:}
\begin{enumerate}
\item Показано, что в комплексной (пылевой) плазме реализуются различные сценарии в зависимости от величины силы конфайнмента.
\item Балансовый подход позволяет точно и достаточно быстро описать данные сценарии.
\item Обнаружен особый тип динамики системы - странный аттрактор.
\end{enumerate}

\textbf{Методология и методы исследования.} Сформулированные задачи были решены на основе современных методов физики конденсированного состояния, статистической физики, химической физики, физики мягкой материи и компьютерного
моделирования.
Расчеты методом МД выполнены в открытых программных пакетах LAMMPS и HOOMD-Blue. Пост-обработка результатов моделирования и экспериментов выполнена с использованием программных кодов, реализованных на Python и C++/CUDA автором настоящей работы.


\textbf{Достоверность} результатов подтверждается корректностью использования методов физики конденсированного состояния и методов вычислительной физики (в частности методов молекулярной динамики); полученные результаты согласуются с ранее известными результатами, представленными в литературе; результаты моделирования методом молекулярной динамики воспроизводимы и устойчивы к изменениям основных параметров.
Кроме того, достоверность результатов подтверждается согласием результатов, полученных на основе разных подходов, включающих теоретические, вычислительные и экспериментальные.

\textbf{Личный вклад автора} состоит в подготовке программных кодов для проведения расчетов, подготовке и проведении расчетов, в пост обработке результатов моделирования методом молекулярной динамики, в участии в пост-обработке экспериментальных результатов, разработке теоретических моделей, сопоставлении результатов теории, моделирования и экспериментов и последующем анализе и интерпретации результатов. Все основные результаты
получены автором лично, либо при непосредственном участии.

\textbf{Теоретической значимостью} обладает ряд результатов настоящей работы.
В частности, предложенный метод анализа мод в жидкостях с учетом эффектов ангармонизма, позволяет изучать структуру спектров элементарных возбуждений в жидкостях вдали от линии плавления, а следовательно устанавливать новые закономерности связи различных динамических, термодинамических и транспортных свойств жидкостей.
Другим важным результатом является разработанная теория антикроссинга мод в простых жидкостях, которая объясняет структуру дисперсионных зависимостей, их перестройку и формирование гибридных мод, что необходимо для корректного анализа данных экспериментов и моделирования методом МД.
Эти результаты важны для понимания физики жидкостей различной природы,
от простых жидкостей и сжиженных благородных газов и до жидких металлов, молекулярных и комплексных жидкостей, жидких плазм и других родственных конденсированных систем.
Полученные результаты могут оказаться полезными для дальнейшей разработки теории жидкого состояния. Результаты исследования систем с невзаимными взаимодействиями открывают новые перспективы для понимания динамики открытых неравновесных многочастичных систем, а также сценариев диссипативных фазовых переходов, наблюдаемых в них.
Из-за широкой распространенности подобных систем в природе, данные результаты могут оказаться полезными для междисциплинарных исследований на стыке таких областей как физика, химия, биология, физика мягкой материи, изучение коллективных явлений в мультиагентных системах и активной материи.

\textbf{Практическая значимость.} С практической точки зрения ценностью обладают метод анализа дисперсионных зависимостей в простых жидкостях и балансовый подход к расчету стационарных состояний систем с невзаимными эффективными взаимодействиями. Последний позволяет выполнять построения петель гистерезиса и диссипативных фазовых диаграмм в системах с невзаимными взаимодействиям с существенно более низкими вычислительными затратами, чем при прямом МД моделировании.
Практической ценностью обладает предложенная модель взаимодействий в комплексной (пылевой) плазме, которая позволяет выполнять моделирование методом МД данной системы и воспроизводить все основные явления в экспериментах с комплексной (пылевой) плазмой, связанные с активационным тепловым поведением.

\textbf{Результат работы} представляет собой решение актуальной задачи физики конденсированного состояния -- разработки новых подходов к анализу спектров элементарных возбуждений в простых жидкостях и экспериментах с кинетическим уровнем разрешения.

\textbf{Апробация работы.} Основные результаты работы были представлены на следующих конференциях, симпозиумах и семинарах:
\begin{enumerate}\itemsep0em
\item Международная конференция «Collective dynamics and pair correlations in atomic and colloidal systems across coupling regimes» (UK, London, 2019);
\item Международный семинар <<Фундаментальные и прикладные проблемы фотоники и физики конденсированного состояния>> (МГТУ им. Н.Э. Баумана, Москва, 2019);
\item Всероссийская конференция <<Проблемы физики твердого тела и высоких давлений>> (ИФВД им. Л.Ф. Верещагина РАН, ФИАН, пос. Вишневка, 2018, 2019);

\end{enumerate}

Отдельные результаты работы нашли отражение в учебной дисциплине, читаемой студентам МГТУ им. Н.Э. Баумана: <<Физические процессы в микроструктурах>>.



\textbf{Публикации.} %Основные результаты диссертационной работы опубликованы в 22 научных работах %\cite{10.1063/1.4926945, 10.1088/0953-8984/28/23/235401, 10.1063/1.5022969, 10.1039/C7SM02429K, 10.1063/1.4921223, 10.1063/1.4979325, 10.1103/PhysRevE.97.022616, 10.1021/acs.langmuir.6b01644, 10.1038/s41598-017-14001-y, 10.1021/acs.jpcc.7b09317, 10.1103/PhysRevE.96.043201, 10.1103/PhysRevLett.121.075003, 10.1063/1.5050708, 10.1039/c8sm01538d, 10.3367/ufnr.2019.01.038520, 10.1063/1.5088141, 10.1063/1.5082785, 10.1117/12.2272954, 10.1364/FIO.2016.FTh4C.2, 10.1039/c8sm01836g, 10.1088/1742-6596/584/1/012025, 10.1088/1742-6596/1135/1/012093}
%(... индексируются в Scopus, ... -- в Web of Science),
%из которых 18 %\cite{10.1063/1.4926945, 10.1088/0953-8984/28/23/235401, 10.1063/1.5022969, 10.1039/C7SM02429K, 10.1063/1.4921223, 10.1063/1.4979325, 10.1103/PhysRevE.97.022616, 10.1021/acs.langmuir.6b01644, 10.1038/s41598-017-14001-y, 10.1021/acs.jpcc.7b09317, 10.1103/PhysRevE.96.043201, 10.1103/PhysRevLett.121.075003, 10.1063/1.5050708, 10.1039/c8sm01538d, 10.1039/c8sm01836g, 10.3367/ufnr.2019.01.038520, 10.1063/1.5088141, 10.1063/1.5082785}
%– научные статьи в журналах, рекомендованных ВАК РФ для публикации основных результатов научных работ соискателей ученой степени доктора наук,
%4 -- публикации в трудах конференций (индексируются в Scopus/WoS).
Основные результаты работы опубликованы в 15 научных работах в журналах, рекомендованных ВАК РФ для публикации основных результатов научных работ, в том числе индексируются в Scopus / Web of Science).


Среди научных изданий, в которых опубликованы результаты работы~-- ведущие мировые журналы (входящие в Q1, WoS/Scopus), как
Physical Review Letters \cite{10.1103/PhysRevLett.121.075003}, The Journal of Chemical Physics \cite{10.1063/1.4926945, 10.1063/1.5022969, 10.1063/1.4921223, 10.1063/1.4979325, 10.1063/1.5050708, 10.1063/1.5088141, 10.1063/1.5082785}, Physical Review E
\cite{10.1103/PhysRevE.97.022616, 10.1103/PhysRevE.96.043201}, Soft Matter \cite{10.1039/C7SM02429K, 10.1039/c8sm01836g}, %Journal of Physical Chemistry C \cite{10.1021/acs.jpcc.7b09317},
%Scientific Reports \cite{10.1038/s41598-017-14001-y},
Langmuir \cite{10.1021/acs.langmuir.6b01644}, Journal of Physics-Condensed Matter \cite{10.1088/0953-8984/28/23/235401}.

О высоком интересе научного сообщества и актуальности результатов работы свидетельствует то, что
статья \cite{10.1063/1.4979325} вошла в коллекцию <<Editors’ Choice 2017>> The Journal of Chemical Physics.
%статья \cite{10.1038/s41598-017-14001-y} стала одной из <<Top 100 Read Articles 2017>> Scientific Reports (Топ 100 самых читаемых статей 2017 года).
Всего соискатель имеет 22 научные публикации, индексируемые в Scopus и Web Of Science.

\textbf{Структура и объем работы.} Научно квалификационная работа состоит из введения, 4 глав и заключения, содержит 116 страниц, 23 рисунков, 1 таблицу.
Список литературы включает 216 источников.

Во \textbf{введении} кратко обосновывается актуальность работы, формулируется цель, перечисляются положения, выносимых на защиту, указывается научная новизна, достоверность, фундаментальная и практическая значимость результатов работы, личный вклад автора, апробация работы и содержание по главам.

\textbf{Глава 1} является обзорной.
В разделе \ref{C1_phonon} кратко рассматриваются коллективные возбуждения в кристаллах, гармоническое приближение для расчета дисперсионных зависимостей и термодинамических свойств кристаллов.
Кратко излагается интерполяционный метод кратчайших графов, демонстрирующий связь спектров возбуждений с парными корреляционными функциями классических кристаллов.
В разделе \ref{C1_liq} рассматриваются коллективные возбуждения в жидкостях. Кратко излагается один из теоретических подходов к расчету дисперсионных зависимостей в простых жидкостях.
В разделе \ref{C1_exp} кратко рассматриваются экспериментальные подходы к изучению спектров элементарных возбуждений в конденсированных системах.
В завершении главы формулируются цель и задачи работы.


\textbf{Глава 2} посвящена изучению элементарных возбуждений в простых двумерных и трехмерных жидкостях в широком диапазоне параметров.
В разделе \ref{C2_MD} приводится описание проводимых расчетов методом МД.
В разделе \ref{C2_disp} анализируется вопрос точности восстановления дисперсионных зависимостей в жидкостях, рассматривается проблема измерения размеров области обратного пространства, соответствующей неустойчивым поперечным модам.
В разделе \ref{C2_ant} излагается теория антикроссинга мод в простых жидкостях.
В разделе \ref{C2_rez_MD} систематически изучаются коллективные возбуждения в различных двумерных и трехмерных жидкостях на основе моделирования методом МД.
Особое внимание уделяется изучению зависимости размеров области в обратного пространстве, соответствующей неустойчивым поперечным модам, от параметра неидеальности.
Полученные результаты обсуждаются в контексте изучения перехода от состояний с жидко-подобной к газо-подобной динамике, который известен как кроссовер Френкеля.
В разделе \ref{C2_rez} обобщаются основные результаты главы.

\textbf{Глава 3} посвящена рассмотрению комплексной пылевой плазмы, как экспериментальной системы, допускающей измерение спектров элементарных возбуждений в жидкостях.
В разделе \ref{C3_NRS} рассматриваются особенности динамики двумерных систем с невзаимными взаимодействиями.
В заключении раздела обсуждается, каким образом полученные результаты объясняют ряд особенностей, наблюдаемых в экспериментах с комплексной (пылевой) плазмой в земных условиях.
В разделе \ref{C3_PU} предлагается простая модель взаимодействий пылевых частиц в комплексной пылевой плазме, позволяющая выполнять моделирование жидких состояний в этой системе методами МД.
В разделе \ref{C3_rez} обобщаются основные результаты главы.

В \textbf{общих выводах и заключении} обобщаются основные результаты работы.
%\newpage